\chapter{Introduction}

News is an important medium through which the members of society get to understand the world. However, despite the normative expectation of neutrality and objectivity \cite{schudson2001objectivity}, news articles covering the same event can vary among different news agencies. The same story can be covered from different perspectives, conventionally referred to as ``frames'' \cite{entman1993,entman2007,entman2010media}. The act ``to frame'' is ``to select some aspects of a perceived reality and make them more salient in a communicating text'' \cite{entman1993}. As a linguistic technique, the practice of framing reflects the underlying preferences and intents of the covering news agency and is known to affect public opinion and political processes \cite{chong2007framing,iyengar1990framing,mccombs2002agenda,price2005framing,rugg1941experiments,schuldt2011global,baumgartner2008decline,dardis2008media,hamborg2020media,drakulich2015explicit,sap2019social}.

Prior work has proposed computational approaches for news framing analysis. One line of work focuses on recovering the underlying bias of the news agencies (\ie, Political Viewpoint Identification) \cite{PVI-survey,li-goldwasser-2021-using,kim-johnson-2022-close,matero-etal-2021-melt-message,li-caragea-2021-multi}, and another line of work aims to locate the framing language and structures in the news articles \cite{hamborg2020media,drakulich2015explicit,sap2019social,baumer-etal-2015-testing}. Recently, there are also efforts paid to generate texts with framing techniques, such as the task of reframing \cite{ziems-etal-2022-inducing,chen-etal-2021-controlled-neural}.

In this work, we propose \texttt{MultiAgencyNews}, a novel large-scale, multi-agency news dataset with crowd-sourced political stances and factuality labels to facilitate framing analysis. Existing news aggregation platforms are utilized to collect metadata for recent social events with news article coverage from multiple news agencies. The metadata includes a series of labels including the news article URLs, news agency names, news agency political stance biases, and news agency factualities. The news articles are filtered with heuristics to balance the labels, guaranteeing that the comparison is fairly conducted.

We continue to propose two different methods of conducting framing analyses on this dataset, the first method, \texttt{SwitchLM}, learns a ``switch'' in a large pre-trained language model's embedding space, the switch shifts the semantic embedding of specific words in a text prompt for text generation. The switch is a learnable parameterized matrix that projects embedding for each word to a linear subspace, where the semantics of corresponding dimensions align with the corresponding directions in a scaled manner.

The second method, \texttt{GenCo}, utilizes generative language models and contrastive learning. An initial configuration vector is inputted to a generator, which is a generative language model such as \texttt{T5}, \texttt{BART}, and \texttt{GPT}. Existing Information Extraction tools are then applied to extract the event structure of the generated news, where two levels of event graphs are involved: 1) the inter-subframe graphs which capture the coarse-grained semantics such as paragraphs, and 2) the intra-subframe graphs which capture the fine-grained semantics such as sentences. The generated event graph is then compared with texts sampled from corresponding news pools from the dataset for contrastive learning with a classifier. The generator and the classifier are collectively trained and mutually enhancing each other.

We further create an interactive demo application with streamlit to directly display results. The application is live and hosted on a cloud server which can be accessed with public IP. The application showcases two functionalities: 1) the \textit{stance-guided generation} functionality, where the user gets to specify a stance on a continuous spectrum which will be used to guide the model on generating with stances, and 2) the \textit{text stance scoring} functionality, where a user-specified piece of news article text can be inferred by a hosted trained model to profile the underlying political bias.